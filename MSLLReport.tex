\documentclass[slidestop,compress,mathserif]{beamer}
\usepackage{setspace}
\usepackage{ragged2e}
\usepackage{biblatex}
\usepackage{algorithm}           
\usepackage{algorithmic}
\bibliography{Domingos}
\let\raggedright=\RaggedRight
%\usepackage[bars]{beamerthemetree} % Beamer theme v 2.2
\usetheme{Antibes} % Beamer theme v 3.0
\usecolortheme{lily} % Beamer color theme
\title{Report}
\author{Cuiyi}
\institute{}

\begin{document}
% coverpage
\begin{frame} % Cover slide
\titlepage
\end{frame}
%section 1 Supervised Learning
\section{Supervised Learning}
%section 1.1 Discriminate Method
    \subsection{Review}
    %review slide
    \begin{frame}[options]
        \frametitle{Discriminate Method Review}

        %\begin{enumerate}
%        \item ...
%        \end{enumerate}
        %\begin{block}<+->{Review}
        \begin{enumerate}
            \item Logistic Regression

                Assume $y=0$ or $y=1$,and $h_w(x)=\frac{1}{1-e^{w^Tx}}$
            \item Perceptron Learning Algorithm

                $h_w(x)$ is a threshold function:

                $$h_w(x) = g_w(w^Tx)=\left\{
                    \begin{array}{ll}
                        1  &\mbox{if $w^Tx \ge 0$}\\
                        -1  &\mbox{otherwise}
                    \end{array}
                    \right.
                $$
        \end{enumerate}
        %\end{block}

        %\begin{figure}
%            \pgfimage[width=5cm ]{"123.jpg"}
%        \end{figure}
    %\subsection{Gradient descent}
    \end{frame}
    %review slide
    \begin{frame}
        \frametitle{Gradient descent Review}
        Batch Gradient descent

            Repeat for each parameter :
            $$w_i := w_i + \alpha \frac{\partial}{\partial w_i}Loss(h_w(x),y)$$
        \begin{figure}
           \pgfimage[width=5cm ]{"image/01.png"}
        \end{figure}



    \end{frame}
    \begin{frame}
        \frametitle{Gradient descent Review(2)}
        \fontsize{8pt}{7.2}\selectfont
        The normal equations:

        For a function $f:\mathbb{R}^{m\times n}\to \mathbb{R}$ define the derivative of $f$ with respect to $A$ to be:

        $$\nabla_Af(A)=\left(
                         \begin{array}{ccc}
                           \frac{\partial}{\partial A_{11}} & \ldots & \frac{\partial}{\partial A_{1n}} \\
                           \vdots & \ddots & \vdots \\
                           \frac{\partial}{\partial A_{m1}} & \ldots & \frac{\partial}{\partial A_{mn}} \\
                         \end{array}
                       \right)
        $$
        For examples, suppose $A=\left(
                                   \begin{array}{cc}
                                     A_{11} & A_{12} \\
                                     A_{21} & A_{22} \\
                                   \end{array}
                                 \right)
        $ is a 2-by-2 matrix,and:
        $$f(A)= \frac{3}{2}A_{11}+5A_{12}^2+A_{21}A_{22}$$
        Then we have:
        $$\nabla_Af(A)=\left(
                         \begin{array}{cc}
                           \frac{3}{2} & 10A_{12} \\
                           A_{22} & A_{21} \\
                         \end{array}
                       \right)
        $$
        Define of \textbf{trace} operator. For an n-by-n matrix A:
        $$trA=\sum_{i=1}^n A_{ii}$$
    \end{frame}
    \begin{frame}
        \frametitle{Gradient descent Review(3)}
        For two matrices $A$ and $B$,such that $ABC$ is square,we have:
        $$trAB=trBA$$
        As corollaries:
        $$trABC=trCAB=trBCA$$
        Here $A$ and $B$ is square matrices and a is real number:
        \begin{eqnarray*}
            trA &=& trA^T\\
            tr(A+B)&=& trA+trB\\
            traA &=& atrA
        \end{eqnarray*}

    \end{frame}
    \begin{frame}
        \frametitle{Gradient descent Review(4)}
        Some facts without proof:
        \begin{eqnarray*}
            \nabla_AtrAB &=& B^T\\
            \nabla_{A^T}f(A) &=& (\nabla_Af(A))^T\\
            \nabla_AtrABA^TC &=& CAB+ C^TAB^T\\
        \end{eqnarray*}
        Revisit Least Squares:

        define \textbf{design matrix}$X$ to be the m-by-n matrix:
        $$X=\left(
              \begin{array}{ccc}
                x_1^{(1)} & \ldots & x_n^{(1)} \\
                \vdots & \ddots & \vdots \\
                x_1^{(m)} & \ldots & x_n^{(m)} \\
              \end{array}
            \right)
        $$
    \end{frame}
    \begin{frame}
        \frametitle{Gradient descent Review(5)}
        define $\vec{y}$ be m-dimensional vector:
        $$\vec{y}=\left(
                               \begin{array}{c}
                                 y^{(1)} \\
                                 y^{(2)} \\
                                 \vdots \\
                                 y^{(m)} \\
                               \end{array}
                             \right)
        $$
        Since $h_w(x^{(i)})=(x^{(i)})^Tw$:
        \begin{eqnarray*}
            Xw-\vec{y} &=& \left(
                         \begin{array}{c}
                           (x^{(1)})^Tw \\
                           \vdots \\
                           (x^{(m)})^Tw \\
                         \end{array}
                       \right)
                       -
                       \left(
                         \begin{array}{c}
                           y^{(1)} \\
                           \vdots \\
                           y^{(m)} \\
                         \end{array}
                       \right)\\
                       &=& \left(
                             \begin{array}{c}
                               h_w(x^{(1)})-y^{(1)} \\
                               \vdots \\
                               h_w(x^{(m)})-y^{(m)} \\
                             \end{array}
                           \right)
        \end{eqnarray*}

    \end{frame}
    \begin{frame}[options]
        \frametitle{Gradient descent Review(5)}
        %\fontsize{8pt}{7.2}\selectfont
        Using the face that for $\vec{z}$,we have $\vec{z}^T\vec{z}=\Sigma_i\vec{z}_i^2$:
        \begin{eqnarray*}
            \frac{1}{2}(Xw-\vec{y})^T(Xw-\vec{y}) &=& \frac{1}{2}\sum_{i=1}^m(h_w(x^{(i)})-y^{(i)})^2\\
            &=& J(w)
        \end{eqnarray*}
        \begin{eqnarray*}
            \nabla_wJ(w) &=& \nabla_w \frac{1}{2}(Xw-\vec{y})^T(Xw-\vec{y})\\
            &=& \frac{1}{2}\nabla_wtr(w^TX^TXw-w^TX^T\vec{y}-\vec{y}^TXw+\vec{y}^T\vec{y})\\
            &=& \frac{1}{2}\nabla_w(trw^TX^TXw-2tr\vec{y}^TXw)\\
            &=& \frac{1}{2}(X^TXw+X^TXw-2X^Tw)\\
            &=& X^TXw-X^T\vec{y}
        \end{eqnarray*}
    \end{frame}
    \subsection{Generative Learning algorithms}
    \begin{frame}[shrink]
        %\fontsize{12pt}{7.2}\selectfont
        \frametitle{What are Generative Learning algorithms}
        \begin{itemize}
          \item Consider a classification problem: distinguish between benign tumors and malignant tumors.
          \item Given a training set, an algorithm like logistic regression or
              the perceptron algorithm (basically) tries to find a straight line that is, a decision boundary—that separates the benign tumor and malignant tumor.
			  Algorithms that try to learn $P(y|x)$ directly are called \textbf{discriminative learning algorithms}.
          \item Different approach: Fist we can build a model of what benign tumors look like. Then we build a seperate model of what malignant tumors look like. Finally, to classify a new sample,we can match the new sample with each model,to see whether one model matches better than one other model.

              Algorithms that try to learn $P(x|y)$(and $P(y)$) are called \textbf{generative learning algorithms}.
              
        \end{itemize}
        $$$$
    \end{frame}
    \begin{frame}
        \frametitle{The multivariate normal distribution}
        Suppose these is a multivariate normal distribution is parameterized by a \textbf{mean vector} $\mu \in \mathbb{R}^n$ and a \textbf{covariance matrix} $\Sigma \in \mathbb{R}^{n\times n}$,its density is given by:
        $$p(x;\mu,\Sigma)=\frac{1}{(2\pi)^{n/2}|\Sigma|^{1/2}}\exp\bigg(-\frac{1}{2}(x-\mu)^T\Sigma^{-1}(x-\mu)\bigg)$$
        \begin{figure}
           \pgfimage[height=3cm ]{"image/02.png"}
        \end{figure}
    \end{frame}
    \begin{frame}
        \frametitle{The multivariate normal distribution(2)}
        \begin{figure}
           \pgfimage[height=3.3cm ]{"image/03.png"}
        \end{figure}
        \begin{figure}
           \pgfimage[height=3cm ]{"image/04.png"}
        \end{figure}
    \end{frame}
    \begin{frame}[shrink]
        \frametitle{The Gaussian Discriminant Analysis model(1)}
        GDA Model is:
        \begin{eqnarray*}
            y &\sim& Bernoulli(\phi)\\
            x|y=0 &\sim& \mathcal{N}(\mu_0,\Sigma)\\
            x|y=1 &\sim& \mathcal{N}(\mu_1,\Sigma)
        \end{eqnarray*}
        Writing out the distribution:
        $$p(y)=\phi^y(1-\phi)^y$$
        $$p(x|y=0) = \frac{1}{(2\pi)^{n/2}|\Sigma|^{1/2}}\exp\bigg( -\frac{1}{2}(x-\mu_0)^T\Sigma^{-1}(x-\mu_0)\bigg)$$
        $$p(x|y=1) = \frac{1}{(2\pi)^{n/2}|\Sigma|^{1/2}}\exp\bigg( -\frac{1}{2}(x-\mu_1)^T\Sigma^{-1}(x-\mu_1)\bigg)$$

    \end{frame}
    \begin{frame}[shrink]
        \frametitle{The Gaussian Discriminant Analysis model(2)}
        The log-likelihood is:
        \begin{eqnarray*}
            \l(\phi,\mu_0,\mu_1,\Sigma) &=& \log\prod_{i=1}^mp(x^{(i)},y^{(i)};\phi,\mu_0,\mu_1,\Sigma)\\
            &=& \log \prod_{i=1}^mp(x^{(i)}|y^{(i)};\mu_0,\mu_1,\Sigma)p(y^{(i)};\phi)
        \end{eqnarray*}
        By maximizing $l$ with respect to the parameters,we find the maximum likelihood estimate of parameters to be:
        \begin{eqnarray*}
            \phi &=& \frac{1}{m}\sum_{i:y^{(i)}=1}^my^{(i)}\\
            \mu_0 &=& \frac{\sum_{i:y^{(i)}=0}^mx^{(i)}}{\sum_{i:y^{(i)}=0}^my^{(i)}}\\
            \mu_1 &=&
            \frac{\sum_{i:y^{(i)}=1}^mx^{(i)}}{\sum_{i:y^{(i)}=1}^my^{(i)}}\\
            \Sigma &=& \frac{1}{m}\sum_{i=1}^m(x^{(i)}-\mu_{y^{(i)}})(x^{(i)}-\mu_{y^{(i)}})^T
        \end{eqnarray*}
    \end{frame}
    \begin{frame}[shrink]
        \frametitle{The Gaussian Discriminant Analysis model(3)}
        Pictorially, what the algorithm is doing can be seen in as follows:
        \begin{figure}
           \pgfimage[height=6cm ]{"image/05.png"}
        \end{figure}
    \end{frame}
    \begin{frame}[shrink]
        \frametitle{Discussion: GDA and logistic regression}
        If we view $p(y=1|x;\phi,\mu_0,\mu_1,\Sigma)$ as a function of $x$, there is a fact that:
        $$p(y=1|x;\phi,\Sigma,\mu_0,\mu_1)=\frac{1}{1+\exp(-w^Tx)}$$
        This is exactly the form that logistic regression used to model $p(y=1|x)$.

        In fact in binary division problem:
            $$p(x|y=1)\sim ExpFamily(x,\eta)\Rightarrow p(y=1|x)=Sigmoid $$

        When would we prefer one model over another?

        \begin{itemize}
          \item GDA makes stronger modelling assumptions, and is more data efficient (i.e., requires less training data to learn “well”) when the modeling assumptions are correct or at least approximately correct.
          \item Logistic regression makes weaker assumptions, and is significantly more robust to deviations from modeling assumptions.
        \end{itemize}

    \end{frame}
    \subsection{Bayesian Decision Theory}
    \begin{frame}
    	\frametitle{Bayes' formula}
    	Suppose that we have the prior probability $P(\omega_j)$
    	and the conditional densities $p(x|\omega_j)$, the posterior
    	probability can be written as:
    	$$P(\omega_j|x)=\frac{p(x|\omega_j)P(\omega_j)}{p(x)},$$ 
    	where
    	$$p(x)=\sum_{j=1}^cp(x|\omega_j)P(\omega_j).$$
    	Bayes' formula can be expressed informally in English:
    	$$posterior = \frac{likelihood\times prior}{evidence}.$$
    \end{frame}
    \begin{frame}
    	\frametitle{Bayesian Decision Theory – Continuous Features}
    	Let $\omega_1,\ldots,\omega_c$ be the finite set of $c$ states of
    	nature ("categories") and $\alpha_1,\ldots,\alpha_a$ be the
    	finite set of a possible action. The Loss function $\lambda
    	(\alpha_i|\omega_j)$ describe the loss incurred for taking
    	action $\alpha_i$ when the state of nature is $\omega_j$. The
    	posterior probability $P(\omega_j|x)$ can be compute from 
    	$P(\omega_j|x)$ by Bayes'formula:
    	$$P(\omega_j|x)=\frac{p(x|\omega_j)P(\omega_j)}{p(x)}$$
    	where the evidence is now
    	$$p(x)=\sum_{j=1}^cp(x|\omega_j)P(\omega_j)$$
    	
    \end{frame}
    \begin{frame}
    	\frametitle{Bayesian Decision Theory – Continuous Features(2)}
    	Suppose that we observe a particular $x$ and that we 
    	contemplate taking action $\alpha_i$.
    	$$R(\alpha_i|x)=\sum_{j=1}^c\lambda(\alpha_i|\omega_j)
    	P(\omega_j|x)$$
    	In decision-theoretic terminology, an expected loss is called a risk,
    	and $R(\alpha_i|x)$ is called the \textbf{conditional risk}.
    	If we define loss function to be
    	$$\lambda(\alpha_i|\omega_j)=\left\{\begin{array}{ll}
    	0 & i=j\\
    	1 & i\neq j
    	\end{array}\right. i,j=1,\ldots,c.$$
    	Condition risk will be
    	$$R(\alpha_i|x) = \sum_{j=1}^c\lambda(\alpha_i|\omega_j)P(\omega_j|x)=1-P(\omega_i|x)$$	
    \end{frame}
    \begin{frame}
    	\frametitle{Classifiers, Discriminant Functions and Decision
    		Surfaces}
    	For 0-1 Loss function case, we can simplely let the discriminant functions to be $g_i(x)=P(\omega_i|x)$. More generally, if we replace every $g_i(x)$ by $f(g_i(x))$, where $f(\cdot)$ is a monotonically increasing fuction, the resulting classification is unchanged.
    	$$g_i(x)=P(\omega_i|x)=\frac{p(x|\omega_i)P(\omega_i)}{\sum_{j=1}^cp(x|\omega_j)P(\omega_j)}
    	$$
    	$$g_i(x) = p(x|\omega_i)P(\omega_i)$$
    	$$g_i(x) = \log p(x|\omega_i)+\log P(\omega_i)$$
    \end{frame}
    \begin{frame}[shrink]
    	\frametitle{Discriminant Functions for the Normal Density}
    	if $p(x|\omega_i) \sim \mathcal{N}(\mu_i,\Sigma_i)$,
    	$$g_i(x)=-\frac{1}{2}(x-\mu_i)^T\Sigma_i^{-1}(x-\mu_i)-\frac{d}{2}\log 2\pi - \frac{1}{2}\log |\Sigma_i|+\log P(\omega_i)$$
    	Case: $\Sigma_i = \sigma^2I$
    	
    	Ignore terms which are independent of i, we obtain:
    	\begin{eqnarray*}
    		g_i(x) &=& -\frac{\|x-\mu_i\|^2}{2\sigma^2}+\log P(\omega_i)\\
    		&=& -\frac{1}{2\sigma^2}(x^Tx-2\mu_i^Tx+\mu_i^T\mu_i)+\log P(\omega_i)
    	\end{eqnarray*}
    	Since $x^Tx$ is the same for all $i$'s, we obtain
    	$$g_i(x) = w_i^Tx+w_{i0}$$
    	where
    	$$w_i=\frac{1}{\sigma^2}\mu_i,\quad w_{i0} = -\frac{1}{2\sigma^2}\mu_i^T\mu_i+\log P(\omega_i)$$
    \end{frame}
    \begin{frame}
    	\frametitle{Discriminant Functions for the Normal Density(2)}
    	For two class, we can set $g_i(x)=g_j(x)$:
    	$$w^T(x-x_0)=0,$$
    	where,
    	$$w = \mu_i-\mu_j$$
    	and,
    	$$x_0=\frac{1}{2}(\mu_i+\mu_j)-\frac{\sigma^2}{\|\mu_i-\mu_j\|}\log\frac{P(\omega_i)}{P(\omega_j)}(\mu_i-\mu_j)$$
    \end{frame}
    \begin{frame}
    	\frametitle{Discriminant Functions for the Normal Density(3)}
    	Case $\Sigma_i=\Sigma$:
    	$$g_i(x) = w^Tx + w_0$$
    	where
    	$$w_i=\Sigma^{-1}\mu_i,\quad w_{i0}=-\frac{1}{2}\mu_i^T\Sigma^{-1}\mu_i+\log P(\omega_i)$$
    	Case $\Sigma_i=?$
    	$$g_i=x^TW_ix+w_i^Tx+w_{i0}$$
    	where,
    	$$W_i=-\frac{1}{2}\Sigma_i^{-1},\quad w_i=\Sigma_i^{-1}\mu_i$$
    	and,
    	$$w_{i0}=-\frac{1}{2}\mu_i^T\Sigma_i^{-1}\mu_i-\frac{1}{2}\log |\Sigma_i| + \log P(\omega_i)$$
    \end{frame}
    \begin{frame}
    	\frametitle{Bayesian estimation}
    	In Bayesian estimation, we consider parameter $\theta$ to be a random variable. Any information we might have about $\theta$ prior to observing the samples is assumed to be contained in a \textit{known} prior density $p(\theta)$. Observation of the samples converts this to a posterior density $p(\theta|\mathcal{D})$
    	
    	Knowing that our goal is to  compute $p(x|\mathcal{D})$, we do this by:
    	$$p(x|\mathcal{D})=\int p(x,\theta|\mathcal{D})d\theta$$ 
    	then:
    	$$p(x|\mathcal{D}) = \int p(x|\theta)p(\theta|\mathcal{D})d\theta$$
    \end{frame}
    \begin{frame}
    	\frametitle{Bayesian estimation(2)}
    	Suppose $\mu$ is the only unknown parameter, and
    	$$p(x|\mu)\sim \mathcal{N}(\mu,\sigma^2).$$
    	Our prior knowledge we might have about $\mu$ can be expressed by a known prior density $p(\mu)$.
    	$$p(\mu)\sim \mathcal{N}(\mu_0,\sigma_0^2)$$
    	To obtain the posterior probability:
    	\begin{eqnarray*}
    		p(\mu|\mathcal{D}) &=& \frac{p(\mathcal{D}|\mu)p(\mu)}{\int p(\mathcal{D|\mu})p(\mu)}\\
    		&=&\alpha\prod_{k=1}^np(x_k|\mu)p(\mu)
    	\end{eqnarray*}
    \end{frame}
    \begin{frame}[shrink]
    	\frametitle{Bayesian estimation(3)}
    	Since $p(x_k|\mu)\sim\mathcal{N}(\mu,\sigma^2)$ and $p(\mu)\sim\mathcal{N}(\mu_0,\sigma_0^2)$:
    	\begin{figure}
    		\pgfimage[width=9cm]{"image/17.png"}
    	\end{figure}
    	Now we know $p(\mu|\mathcal{D})$ is a Gaussian distribution, If we write $p(\mu|\mathcal{D}\sim\mathcal{N}(\mu_n\sigma_n^2))$,we obtain:
    	$$\mu_n=\left(\frac{n\sigma_0^2}{n\sigma_0^2+\sigma^2}\right)\bar{x_n}+\frac{\sigma^2}{n\sigma_0^2+\sigma^2}\mu_0$$
    	$$\sigma_n^2=\frac{\sigma_0^2\sigma^2}{n\sigma_0^2+\sigma^2}$$
    	
    \end{frame}
    \begin{frame}
    	Now we need to compute the $p(x|\mathcal{D})$:
    	\begin{figure}
    		\pgfimage[width=10cm]{"image/18.png"}
    	\end{figure}
    	Hence,
    	$$p(x|\mathcal{D})\sim\mathcal{N}(\mu_n,\sigma^2+\sigma_n^2)$$
    \end{frame}
    \subsection{Support Vector Machines}
    \begin{frame}[shrink]
        \frametitle{Margin(1)}
        Previous:

        Consider logistic regression,we compute $w^Tx$ then we:
        \begin{itemize}
          \item Predict "1" iff $w^Tx\ge 0$.
          \item Predict "0" iff $w^Tx<0$.
        \end{itemize}
        The larger $w^Tx$ is,the larger also is $h_w(x)=p(y|x;w)$:
        \begin{itemize}
          \item if $w^Tx \gg 0$ then we are very "confident" predicting "1".
          \item if $w^Tx \ll 0$ then we are very "confident" predicting "0".
        \end{itemize}
        \begin{figure}
           \pgfimage[width=6cm,height=3cm]{"image/06.png"}
        \end{figure}



    \end{frame}
    \begin{frame}[shrink]
        \frametitle{Margin(2)}
        To make our discussion of SVMs easier,we need to change our notation:

        we use $y \in {1,-1}$ to denote class label, and change classifier as:
        $$h_{w,b}(x)=g(w^Tx+b)$$
        Here $g$ is a threshold function.

        Given a training example $(x^{(i)},y^{(i)})$,we define \textbf{functional margin} of $(w,b)$ with respect to the training example.
        $$ \hat{\gamma}^{(i)}=y^{(i)}(w^Tx^{(i)}+b) $$
        A large functional margin represents a confident and a correct prediction.But by exploiting our freedom to scale w and b, we can make the functional margin arbitrarily large without really changing anything meaningful.

        Given a training set $S$,we also define the function margin of $(w, b)$ with respect to $S$:

        $$\hat{\gamma}= \min_{i=1,\ldots,m}\hat{\gamma}^{(i)}$$

    \end{frame}
    \begin{frame}[shrink]
        \frametitle{Margin(3)}
        Let’s talk about \textbf{geometric margins}. Consider the picture below:
        \begin{figure}
           \pgfimage[width=5cm,height=3cm]{"image/07.png"}
        \end{figure}
        Suppose A is $x^{(i)}$,then B is $x^{(i)}-\gamma^{(i)}\cdot (w/\|w\|)$,B lies on the decision boundary,Hence:
        $$w^T\bigg(x^{(i)}-\gamma^{(i)}\cdot\frac{w}{\|w\|}\bigg)+b=0$$
        Solving for $\gamma^{(i)}$ yields:
        $$\gamma^{(i)}=\bigg(\frac{w}{\|w\|}\bigg)^Tx^{(i)}+\frac{b}{\|w\|}$$
    \end{frame}

    \begin{frame}
        \frametitle{Margin(4)}
        More generally,define geometric margins of $(w,b)$ with respect to a training example $(x^{(i)},y^{(i)})$ to be
        $$\gamma^{(i)}=y^{(i)}\left(\bigg(\frac{w}{\|w\|}\bigg)^Tx^{(i)}+\frac{b}{\|w\|}\right)$$
        Given a training set $S$, we also define the geometric margin of $(w, b)$ with respect to S to be the smallest of the geometric margins on the individual training examples:
        $$\gamma=\min_{i=1,\ldots,m}\gamma^{(i)}$$
    \end{frame}

    \begin{frame}[shrink]
        \frametitle{The optimal margin classifier}
        Given a training set,We try to find a decision boundary that maximizes the margin. We can pose the following optimization problem
        \begin{eqnarray*}
            \max_{\gamma,w,b} && \gamma\\
            s.t. && y^{(i)}(w^Tx^{(i)}+b)\ge \gamma,i=1,\ldots,m\\
            && \|w\|=1.
        \end{eqnarray*}
        The "$\|w\|$" constraint is a nasty(non-convex) constraint.
        Transform to a nicer one:
        \begin{eqnarray*}
            \max_{\gamma,w,b} && \frac{\hat{\gamma}}{\|w\|}\\
            s.t. && y^{(i)}(w^Tx^{(i)}+b)\ge \hat{\gamma},i=1,\ldots,m
        \end{eqnarray*}
        We get rid of "$\|w\|=1$" which we don't like. But the downside is that we now have a nasty(non-convex) objective function.
        $$$$
    \end{frame}

    \begin{frame}[shrink]
        \frametitle{The optimal margin classifier(2)}
        Since we can add an arbitrary scaling constraint on w and b without changing anything.We introduce a scaling constraint:
        $$\hat{\gamma}=1$$
        Plugging this into our problem above,and noting that maximizing $\hat{\gamma}/\|w\|=1/\|w\|$ is the same thing as minimizing$\|w\|^2$, we now have the following optimization problem:
        \begin{eqnarray*}
            \min_{\gamma,w,b} && \frac{1}{2}\|w\|^2\\
            s.t. && y^{(i)}(w^Tx^{(i)}+b)\ge 1,i=1,\ldots,m
        \end{eqnarray*}
        Now,we have an optimization problem with a convex quadratic objective and only linear constraints.This optimization problem can be solved using commercial quadratic programming (QP) code.
        $$$$
    \end{frame}
    \begin{frame}[shrink]
        \frametitle{Lagrange duality}
        Consider problem of the following form:
        \begin{eqnarray*}
            \min_w && f(w)\\
            s.t. && h_i(w)=0,i=1,\ldots,l.
        \end{eqnarray*}
        Define \textbf{Lagrangian} to be
        $$\mathcal{L}(w,\beta)=f(w)+\sum_{i=1}^l\beta_ih_i(w)$$
        Here,the $\beta_i$'s are called the \textbf{Lagrange multipliers}. Set $\mathcal{L}$'s partial derivatives to zero:
        $$\frac{\partial\mathcal{L}}{\partial w_i}=0;\frac{\partial\mathcal{L}}{\partial \beta_i}=0$$
        We can solve for $w$ and $\beta$
        $$$$
    \end{frame}
    \begin{frame}[shrink]
        \frametitle{Lagrange duality(2)}
        Consider the following, which we call the \textbf{primal} optimization problem:
        \begin{eqnarray*}
            \min_w && f(w)\\
            s.t. && g_i(w) \le 0,i=1,\ldots,k\\
            && h_i(w)=0,i=1,\ldots,l
        \end{eqnarray*}
        Define \textbf{generalized Lagrangian}
        $$\mathcal{L}(w,\alpha,\beta)=f(w)+\sum_{i=1}^k\alpha_ig_i(w)+\sum_{i=1}^l\beta_ih_i(w).$$
        Here the $\alpha_i$'s and $\beta_i$'s are the Lagrange multipliers.Consider the quantity:
        $$\theta_\mathcal{P}=\max_{\alpha,\beta:\alpha_i\ge 0}f(w)+\sum_{i=1}^k\alpha_ig_i(w)+\sum_{i=1}^l\beta_ih_i(w)$$
        Here,the "$\mathcal{P}$" subscript stands for "primal".
        $$$$
    \end{frame}
    \begin{frame}
        \frametitle{Lagrange duality(3)}
        Let some $w$ be given. If $w$ violates any of the primal constraints,then it's easy to verify that $\theta_\mathcal{P}=\infty$. Conversely, if the constraints are indeed satisfied for a particular value of w,then $\theta_\mathcal{P}=f (w)$.Hence,
        $$\theta_\mathcal{P}(w)=\left\{\begin{array}{ll}
                        f(w)  & \mbox{if $w$ satisfied primal conatraints}\\
                        \infty  &\mbox{otherwise}
                        \end{array}
                        \right.
        $$
        If we consider the minimization problem
        $$\min_w\theta_\mathcal{P}(w)=\min_w\max_{\alpha,\beta,\alpha_i\ge 0} \mathcal{L}(w,\alpha,\beta)$$
        It is the same problem of primal problem.Define optimal value of the objective to be $p^*=\min_w\theta_\mathcal{P}(w)$
    \end{frame}
    \begin{frame}
        \frametitle{Lagrange duality(4)}
        Define:
        $$\theta_\mathcal{D}(\alpha,\beta)=\min_w\mathcal{L}(w,\alpha,\beta).$$
        Here,the "$\mathcal{D}$" subscript stands for "dual".

        Now the \textbf{dual} optimization problem:
        $$\max_{\alpha,\beta:\alpha_i \ge 0}\theta_\mathcal{D}(\alpha,\beta)=\max_{\alpha,\beta:\alpha_i \ge 0}\min_w\mathcal{L}(w,\alpha,\beta)$$
        define the optimal value of the dual problems objective to be $d^*=\max_{\alpha,\beta:\alpha_i \ge 0}\theta_\mathcal{D}(\alpha,\beta)$.

        It can easily be shown that
        $$d^*=\max_{\alpha,\beta:\alpha_i\ge 0}\min_w\mathcal{L}(w,\alpha,\beta) \le \min_w\max_{\alpha,\beta:\alpha_i\ge 0}\mathcal{L}(w,\alpha,\beta)=p^*$$
        However under certain conditions,we will have $d^*=p^*$
    \end{frame}
    \begin{frame}[shrink]
        \frametitle{Lagrange duality(5)}
        Let's see what these conditions are:
        \begin{itemize}
          \item $f$ and the $g_i$'s are convex;
          \item $h_i$'s are affine;
          \item $g_i$'s are (strictly) feasible; this means that there exits some $w$ so that $g_i(w)<0$ for all i.
        \end{itemize}
        $w^*$,$\alpha^*$ and $\beta^*$ satisfy the \textbf{Karush-Kuhn-Tucker} (\textbf{KKT}) \textbf{conditions}, which are as follows:
        \begin{eqnarray*}
            \frac{\partial}{\partial w_i}\mathcal{L}(w^*,\alpha^*,\beta^*) &=& 0,i=1,\ldots,n\\
            \frac{\partial}{\partial \beta_i}\mathcal{L}(w^*,\alpha^*,\beta^*)&=& 0,i=1,\ldots,l\\
            \alpha_i^*g_i(w^*) &=& 0,i=1,\ldots,k\\
            g_i(w^*) &\le& 0,i=1,\ldots,k\\
            \alpha^* &\ge& 0,i=1,\ldots,k
        \end{eqnarray*}
    \end{frame}
    \begin{frame}
        \frametitle{The optimal margin classifier(3)}
        we posed the following (primal) optimization
         problem for finding the optimal 
         margin classifier:
         \begin{eqnarray*}
         	\min_{\gamma,w,\beta} && \frac{1}{2}\|w\|^2\\
         	s.t && y^{(i)}(w^Tx^{(i)}+b)\ge 1,i=1,\ldots,m
         \end{eqnarray*}
         We can write the constraints as:
         $$g_i(x)=1-y^{(i)}(w^Tx^{(i)}+1)\le 0,i=1,\ldots,m$$
         Note that from the KKT condition,we will have $\alpha_i>0$
         only for that training example that have function margin 
         exactly equal to one 
    \end{frame}
    \begin{frame}
    	\frametitle{The optimal margin classifier(4)}
    	\begin{figure}
           \pgfimage[width=5cm ]{"image/08.png"}
        \end{figure}
        The points with the smallest margins are exactly
         the ones closest to the decision boundary;here,
          these are the three points which are called 
          the \textbf{support vectors} in this problem.
    \end{frame}
    \begin{frame}[shrink]
    	\frametitle{The optimal margin classifier(5)}
    	Construct Lagrangian for our problem:
    	$$ \mathcal{L}(w,b,\alpha)=\frac{1}{2}
    	\|w\|^2-\sum_{i=1}^m[y^{(i)}(w^Tx^{(i)}+b)-1]$$
    	Note that there are only $\alpha_i$ but no $\beta_i$ 
    	Lagrange multipliers.
    	To minimize $\mathcal{L}(w,b,\alpha)$ for fix $\alpha$:
    	$$\nabla_w\mathcal{L}(w,b,\alpha)=w-\sum_{i=1}^m\alpha_i
    	y^{(i)}x^{(i)}\stackrel{\mathrm{set}}{=}0$$
    	This implies that:
    	\begin{eqnarray}
    		w=\sum_{i=1}^m\alpha_iy^{(i)}x^{(i)}
    	\end{eqnarray}
    \end{frame}
    \begin{frame}[shrink]
    	\frametitle{The optimal margin classifier(6)}
    	As for the derivation with respect to b,we obtain:
    	\begin{equation}
    		\frac{\partial}{\partial b}\mathcal{L}(w,b,\alpha)=
    		\sum_{i=1}^m\alpha_iy^{(i)}=0
    	\end{equation}
    	Plugging w back into the Lagrangian:
    	$$\mathcal{L}(w,b,\alpha)=\sum_{i=1}^m\alpha_i-\frac{1}{2}
    	\sum_{i,j=1}^my^{(i)}y^{(j)}\alpha_i\alpha_j\langle 
    	x^{(i)},x^{(j)}\rangle$$
    	We obtain the following dual problem:
    	\begin{eqnarray*}
    		\max_\alpha && W(\alpha)=\sum_{i=1}^m\alpha_i-
    		\frac{1}{2}\sum_{i,j=1}^my^{(i)}y^{(j)}
    		\alpha_i\alpha_j\langle x^{(i)},x^{(j)}\rangle \\
    		s.t. && \alpha_i \ge 0,\quad i=1,\ldots,m\\
    		&& \sum_{i=1}^m\alpha_iy^{(i)}=0
    	\end{eqnarray*}
    \end{frame}
    \begin{frame}[shrink]
    	\frametitle{The optimal margin classifier(7)}
    	Having found $w^*$,then we can get $b^*$ as:
    	$$b^*=-\frac{\max_{i:y^{(i)}=-1}{w^*}^T+\min_{i:y^{(i)}=1}
    	{w^*}^Tx^{(i)}}{2}$$
    	Suppose we've fit our model's parameters to a training set
    	,and now wish to make a prediction at a new point $x$.We
    	 would calculate $w^Tx+b$,and do prediction.But using (1),
    	 this quantity can also be written:
    	$$w^Tx+b=\sum_{i=1}^m\alpha_iy^{(i)}\langle x^{(i)},x\rangle+b$$
    	Using (2) we saw that the $\alpha_i$'s will all be zero
    	 except for the support vector.   	     	
    \end{frame}
    \begin{frame}
    	\frametitle{Kernels}
    	Define \textbf{Kernel} to be:
    	$$K(x,z)=\phi(x)^T\phi(z)$$
    	we let $\phi$ denote the \textbf{feature mapping}, which 
    	maps from the original input value (\textbf{attributes}) to
    	the features in higher dimensional.
    	
    	Let's see an example. Suppose $x,z\in \mathbb{R}^n$, and
    	consider:
    	$$K(x,z)=(x^Tz)^2=\sum_{i,j=1}^n(x_ix_j)(z_iz_j)=
    	\phi(x)^T\phi(z)$$
    	Thus, we can see feature mapping $\phi$ is given by 
    	(shown here for the case of $n=3$)
    	$$\phi(x)=(x_1x_1,x_1x_2,\ldots,x_3x_3)^T$$
    	
    	 
    \end{frame}
    \begin{frame}[shrink]
    	\frametitle{Kernels(2)}
    	Another example:
    	\begin{eqnarray*}
    		K(x,z) &=& (x^Tz+c)^2\\
    		&=& \sum_{i,j=1}^n(x_ix_j)(z_iz_j)+\sum_{i=1}^n
    		(\sqrt{2c}x_i)(\sqrt{2c}z_i)+c^2
    	\end{eqnarray*}
    	feature mapping:(Again shown for $n=3$)
    	$$\phi(x)=(x_1x_1,x_1x_2,\ldots,x_3x_3,\sqrt{2c}x_1,
    	\sqrt{2c}x_2,\sqrt{2c}x_3,c)^T$$
    	More broadly, the kernel $K(x,z)=(x^Tz+c)^d$ corresponds
    	to a feature mapping to an ${n+d}\choose{d}$ feature space.
    	
    	Map attribute to infinity dimension \textbf{Gaussian kernel}:
    	$$K(x,z)=\exp \left(-\frac{\|x-z\|^2}{2\sigma^2}\right)$$
    \end{frame}
    \begin{frame}
    	\frametitle{Kernels(3)}
    	More broadly, given some function $K$, how can we tell if 
    	it's a valid kernel; i.e., can we tell if there is some 
    	feature mapping $\phi$ so that $K(x,z)=\phi(x)^T\phi(z)$ for
    	all $x$, $z$?
    	\\
    	Consider some finite set of $m$ points $\{x^{(1)},x^{(2)},
    	\ldots,x^{(m)}\}$, and let a m-by-m matrix $K$ be defined
    	as
    	$$K_{ij}=K(x^{(i)},y{(i)})$$
    	$K$ is called the \textbf{Kernel matrix}.
    	
    	Now, if $K$ is a valid Kernel, then 
    	$$K_{ij}=
   	    \phi(x^{(i)})^T\phi(x^{(j)})=\phi(x^{(j)})^T\phi(x^{(i)})
   	    =K_{ji}$$
   	    Hence, $K$ must be symmetric. 
    \end{frame}
    \begin{frame}[shrink]
    	\frametitle{Kernels(4)}
    	 More over, letting $\phi_k(x)$
   	    denote the $k$-th coordinate of the vector $\phi(x)$,we find
   	    that for any vector $z$, we have
   	    \begin{eqnarray*}
   	    	z^TKz &=& \sum_i\sum_jz_iK_{ij}z_j\\
   	    	&=& \sum_i\sum_jz_i\phi(x^{(i)})^T\phi(x^{(j)})z_j\\
   	    	&=& \sum_i\sum_jz_i\sum_k\phi_k(x^{(i)})^T\phi_k(x^{(j)})
   	    	z_j\\
   	    	&=& \sum_k\sum_i\sum_jz_i\phi_k(x^{(i)})^T\phi_k(x^{(j)})
   	    	z_j\\
   	    	&=& \sum_k\left(\sum_iz_i\phi_k(x^{(i)})\right)^2 \ge 0
   	    \end{eqnarray*}
   	    This shows that K is positive semi-definite
   	    $$$$    	
    \end{frame}
    \begin{frame}
    	\frametitle{Kernels(5)}
    	\textbf{Theorem (Mercer).} Let $K:\mathbb{R}^n\times
    	\mathbb{R}^n\rightarrow\mathbb{R}$ be given. Then for
    	$K$ to be a valid (Mercer) Kernel, it is necessary and 
    	sufficient that for any $\{x^{(1)},\ldots,x^{(m)}\},
    	(m<\infty)$,the corresponding kernel matrix is symmetric
    	positive semi-definite.
    \end{frame}
    \begin{frame}[shrink]
    	\frametitle{Regularization and the non-separable case}
    	\begin{figure}
           \pgfimage[width=7cm ]{"image/09.png"}
        \end{figure}
        To make the algorithm work for non-linearly separable 
        datasets as well as be less sensitive to outliers, we
        reformulate our optimization ($l_1$ \textbf{regularization}):
        \begin{eqnarray*}
        	\min_{\gamma,w,b} && \frac{1}{2}\|w\|^2+C\sum_{i=1}^m\xi_i\\
        	s.t && y^{(i)}(w^Tx^{(i)}+b)\ge 1-\xi_i,i=1,\ldots,m\\
        	&& \xi_i \ge 0,i=1,\ldots,m
        \end{eqnarray*}
        
        
    \end{frame}
 	\begin{frame}[shrink]
 		\frametitle{Regularization and the non-separable case(2)}
 		As before, we can form the Lagrangian:
 		$$\mathcal{L}(w,b,\xi,\alpha,\gamma)=\frac{1}{2}w^Tw+
 		C\sum_{i=1}^m\xi_i-\sum_{i=1}^m\alpha_i\big(y^{(i)}
 		(x^Tw+b)-1+\xi_i\big)-\sum_{i=1}^m\gamma_i\xi_i$$
 		Here, the $\alpha_i$'s and $\gamma_i$'s are our
 		Lagrangian multipliers.
 		
 		After some work,we obtain the dual form of the problem:
 		\begin{eqnarray*}
 			\max_\alpha && W(\alpha)=\sum_{i=1}^m\alpha_i-
 			\frac{1}{2}\sum_{i,j=1}^my^{(i)}y^{(j)}\alpha_i
 			\alpha_j\langle x^{(i)},x^{(j)}\rangle \\
 			s.t. && 0\le \alpha_i \ge C,i=1,\ldots,m\\
 			&& \sum_{i=1}^m\alpha_iy^{(i)}=0
 		\end{eqnarray*}
	\end{frame} 
	
	\begin{frame}
		\frametitle{The SMO algorithm}
		Consider trying to solve the 
		unconstrained optimization problem
		$$\max_\alpha W(\alpha_1,\alpha_2,\ldots,\alpha_m).$$
		The algorithm here is called \textbf{coordinate ascent}:
		\begin{figure}
           \pgfimage[width=9cm ]{"image/10.png"}
        \end{figure}
	\end{frame}
	\begin{frame}
		\frametitle{The SMO algorithm(2)}
		\begin{figure}
           \pgfimage[width=7cm ]{"image/11.png"}
        \end{figure}
        The ellipses in the figure are the contours of 
        a quadratic function that we want to optimize.
	\end{frame}
	\begin{frame}
		\frametitle{The SMO algorithm(3)}
		\begin{figure}
           \pgfimage[width=10cm ]{"image/12.png"}
        \end{figure}
	\end{frame}
	\subsection{Learning Theory}
	\begin{frame}
		\frametitle{Bias/variance tradeoff}
		\begin{figure}
           \pgfimage[width=10cm ]{"image/13.png"}
           \begin{itemize}
           	\item The linear model suffers from large bias, and may 
           	underfit the data.
           	\item The 5th order polynomial model suffers from large 
           	variance, and may overfit the data.
           \end{itemize}
        \end{figure}
	\end{frame}	
	\begin{frame}
		\frametitle{Preliminaries}
		\textbf{Lemma.} (The union bound). Let $A_1,A_2,\ldots,A_k$ 
		be $k$ different events (that may not be independent). Then
		$$P(A_1\cup\ldots\cup A_k)\le P(A_1)+\ldots+P(A_k)$$
		\textbf{Lemma.} (Hoeffding inequality) Let $Z_1,\ldots,Z_m$ 
		be the $m$ independent and identically distributed (iid) random
		variables drawn from a $\mathrm{Bernoulli} (\phi)$ distribution.
		I.e., $P(Z_i=1)=\phi$, and $P(Z_i=0)=1-\phi$. Let $\hat{\phi}
		=(1/m)\sum_{i=1}^mZ_i$ be the mean of these random variables,
		and let any $\gamma >0$ be fixed. Then
		$$P(|\phi-\hat{\phi}|>\gamma)\le 2\exp(-2\gamma^2m)$$
		
	\end{frame}
	\begin{frame}
		\frametitle{Preliminaries(2)}
		We assume we are given a training set $S=\{(x^{(i)},y^{(i)});
		i=1,\ldots,m\}$ of size m, where the training examples $(x^{(i)},
		y^{(i)})$ are drawn iid from some probability distribution
		$\mathcal{D}$. For a hypothesis h, we define the \textbf{
		training error} (also called the \textbf{empirical risk} or
		\textbf{empirical error} in learning theory) to be
		$$\hat{\varepsilon}(h)=\frac{1}{m}\sum_{i=1}^m1\{
		h(x^{(i)}) \neq y^{(i)}\}$$
		Here $1\{condition\}$ means, if condition is true then 
		$1\{condition\}=1$, otherwise $1\{condition\}=0$. 
		
		Define \textbf{generalization error} to be
		$$\varepsilon(h)=P_{(x,y)\sim \mathcal{D}}(h(x)\neq y)$$
	\end{frame}
	\begin{frame}[shrink]
		\frametitle{Empirical risk minimization (ERM)}
		We define the \textbf{hypothesis class} $\mathcal{H}$ used 
		by a learning algorithm to be the set of all classifiers
		considered by it. For linear classification, $\mathcal{H}=
		\{h_w:h_w(x)=1\{w^Tx \ge 0\},w\in\mathbb{R}^{n+1}\}$.
		
		Empirical risk minimization can now be thought of as a minimization over
		the class of functions $\mathcal{H}$:
		$$\hat{h}=\mathrm{arg}\min_{h\in\mathcal{H}}\hat{\varepsilon}(h)$$
		Let's consider the case of finite $\mathcal{H}=\{h_1,\ldots,h_k\}$
		. Take any one, fixed, $h_i\in \mathcal{H}$. Consider a Bernoulli
		random variable $Z_j=1\{h_i(x^{(i)})\neq y^{(j)}\}$, then the
		training error can be written
		$$\hat{\varepsilon}(h_i)=\frac{1}{m}\sum_{j=1}^mZ_j$$
		Since our training set was drawn idd from $\mathcal{D}$
		$$\varepsilon(h_i)=E[Z_j]=P(Z_j=1)$$
	\end{frame}	
	\begin{frame}[shrink]
		\frametitle{Empirical risk minimization (ERM)(2)}
		We can apply the Hoeffding inequality, and obtain
		$$P(|\varepsilon(h_i)-\hat{\varepsilon}(h_i)|>\gamma) \le 
		2\exp(-2\gamma^2m)$$
		Let $A_i$ denote the event that $|\varepsilon(h_i)-\hat{
		\varepsilon(h_i)}|>\gamma$. 
		Using the union bound, we have
		\begin{eqnarray*}
			P(\exists h \in \mathcal{H}.|\varepsilon(h_i)-
			\hat{\varepsilon}(h_i)|>\gamma) &=& P(A_1 \cup \ldots \cup
			A_k)\\
			&\le& \sum_{i=1}^kP(A_i)\\
			&\le& \sum_{i=1}^k2\exp(-2\gamma^2m)\\
			&=& 2k\exp(-2\gamma^2m)
		\end{eqnarray*}
		
	\end{frame}	
	\begin{frame}[shrink]
		\frametitle{Empirical risk minimization (ERM)(3)}
		Subtract both sides from 1, we obtain
		\begin{eqnarray*}
			P(\neg\exists h \in \mathcal{H}.|\varepsilon(h_i)-
			\hat{\varepsilon}(h_i)|>\gamma) &=& P(\forall \in \mathcal{H}.
			|\varepsilon(h_i)-\hat{\varepsilon}(h_i)|\le \gamma)\\
			&\ge& 1-2k\exp(-2\gamma^2m)
		\end{eqnarray*}
		So, with probability at least $1-2k\exp(-2\gamma^2m)$ we have
		that $\varepsilon(h)$ will be within $\gamma$ of $\hat{\varepsilon(h)}$
		for all $h \in \mathcal{H}$. This is called a \textbf{uniform
		convergence result}.
		
		Given $\gamma$ and some $\delta >0$, by setting $\delta = 
		2k\exp(-2\gamma^2m)$,we find that if
		$$m \ge \frac{1}{2\gamma^2}\log\frac{2k}{\delta},$$
		then with probability at least $1-\delta$, we have that
		$|\varepsilon(h)-\hat{\varepsilon}(h)| \le \gamma$ for all
		$h \in \mathcal{H}$.
		
		The training set size $m$ that a certain method or algorithm
		requires in order to achieve a certain level if performance
		is also called the algorithm's \textbf{sample complexity}.
		$$$$		
	\end{frame}
	\begin{frame}
		\frametitle{Empirical risk minimization (ERM)(4)}
		Similarly,we can also hold $m$ and $\delta$ fixed and solve
		for $\gamma$.
		$$|\hat{\varepsilon(h)}-\varepsilon(h)| \le \gamma \le \sqrt{
		\frac{1}{2m}\log\frac{2k}{\delta}}$$
		Define $h^*=\mathrm{arg}\min_{h\in \mathcal{H}}\varepsilon(h)$.
		We have:
			$$\varepsilon(\hat{h}) \le \hat{\varepsilon}(\hat{h})+\gamma
			\le \hat{\varepsilon}(h^*)+\gamma \le \varepsilon(h^*)+2\gamma $$ 
		\textbf{Theorem.} Let $|\mathcal{H}|=k$, and let any $m$, 
		$\delta$ be fixed. Then with probability at least $1-\delta$,
		we have that
		$$\varepsilon(\hat{h})\le \left(\min_{h \in \mathcal{H}}
		\varepsilon(h)\right)+2\sqrt{\frac{1}{2m}\log\frac{2k}{\delta}}$$
	\end{frame}
	\begin{frame}
		\frametitle{Empirical risk minimization (ERM)(5)}
		Given a set $S=\{x^{(1)},\ldots,x^{(d)}\}$, we say that $\mathcal{H}$
		\textbf{shatters} $S$ if $\mathcal{H}$ can realize any labelling
		on $S$. I.e., if for any set of labels $\{y^{(1)},\ldots,
		y^{(d)}\}$, there exists some $h\in\mathcal{H}$ so that
		$h(x^{(i)})=y^{(i)}$ for all $i=1,\ldots,d$. 
		
		Given a hypothesis class $\mathcal{H}$, we can define its
		\textbf{Vapnik- Chervonenkis dimension}, written $VC(\mathcal{H})$,
		to be the size of the largest set that is shattered by $\mathcal{H}$.
		
		\textbf{Theorem.} Let $\mathcal{H}$ be given, and let $d = 
		VC(\mathcal{H})$. Then with probability at least $1-\delta$, 
		we have that for all $h\in\mathcal{H}$,
		$$|\epsilon(h)-\hat{\epsilon}(h)| \le O\left(\sqrt{
		\frac{d}{m}\log{\frac{m}{d}}+\frac{1}{m}\log\frac{1}{\delta}}
		\right).$$
	\end{frame}	
	\begin{frame}
		\frametitle{Empirical risk minimization (ERM)(6)}
		Thus, with probability at least $1-\delta$, we also have that:
		$$\epsilon(\hat{h})\le \epsilon(h^*)+O\left(\sqrt{
		\frac{d}{m}\log{\frac{m}{d}}+\frac{1}{m}\log\frac{1}{\delta}}
		\right).$$
		\textbf{Corollary.} For $|\epsilon(h)-\hat{\epsilon}(h)|\le 
		\gamma$ to hold all $h\in\mathcal{H}$ with probability at 
		least $1-\delta$, it suffices that $m=O_{\gamma,\delta}(d)$.
	\end{frame}						   
	\section{Unsupervised Learning}
	\begin{frame}
		In Mixture densities we have some assumptions.
		\begin{itemize}
			\item[1.] The samples come from a known number of classes.
			\item[2.] The prior probabilities $P(\omega_j)$ for each class are known, $j=1,\ldots,c$.
			\item[3.] The forms for the class-conditional probability densities $p(x|\omega_j,\theta_j)$ are known, $j=1,\ldots,c$.
			\item[4.] The values for the c parameter vectors. $\theta_1,\ldots,\theta_c$ are unknown.
			\item[5.] The category labels are unknown.
		\end{itemize}
		Samples are assumed to be obtained by selecting state if nature $\omega_j$ with probability $P(\omega_j)$ and then selecting an x according to the probability law $p(x|\omega_j,\theta_j)$. Thus probability density function for the samples is given by
		$$p(x|\theta)=\sum_{j=1}^{c}p(x|\omega_j,\theta_j)P(\omega_j)$$
	\end{frame}
	\begin{frame}
		\textbf{Definition}: A density $p(x|\theta)$ is said to be \textit{identifiable} if $\theta \neq \theta'$ implies that there exists an $x$ such that $p(x|\theta) \neq p(x|\theta')$. 
		
		Or put another way, a density $p(x|\theta)$ is \textit{not identifiable} if we cannot recover a unique $\theta$.
		
		A example for unidentifiable:
		\begin{eqnarray*}
			P(x|\theta) &=& \frac{1}{2}\theta_1^x(1-\theta_1)^{1-x}+\frac{1}{2}\theta_2^x(1-\theta_2)^{1-x}\\
			&=& \left\{\begin{array}{ll}
				\frac{1}{2}(\theta_1+\theta_2) & \mbox{if $x=1$;}\\
				1-\frac{1}{2}(\theta_1+\theta_2) & 	\mbox{if $x=0$.}
			\end{array}
			\right.
		\end{eqnarray*}
		Suppose, for example, that we know for our data that P(x = 1|θ) = 0.6, and hence that $P(x=0|\theta)=0.4$. Then we know the function $P(x|\theta)$, but we cannot determine $\theta$, and hence cannot extract the component distributions. The most we can say is that $\theta_1+\theta_2=1.2$. Thus, here we have a case in which the mixture distribution is completely unidentifiable.
	\end{frame}
	\begin{frame}
		Suppose now that we are given a set $\mathcal{D}=\{x_1,\ldots,x_n\}$ of $n$ unlabeled samples drawn independently from mixture density.
		$$p(x|\theta) = \sum_{j=1}^cp(x|\omega_j,\theta_j)P(\omega_j)$$
		where the full parameter vector $\theta$ is fixed but unknown. The likelihood of the observed samples is, by definition, the joint density
		$$p(\mathcal{D}|\theta)\equiv\prod_{k=1}^np(x_k|\theta)$$
		The maximum-likelihood estimate $\hat{\theta}$ is that value of $\theta$ that maximizes $p(\mathcal{D}|\theta)$.
	\end{frame}
	\begin{frame}
		The Log maximum-likelihood:
		$$l=\sum_{k=1}^n\log p(x_k|\theta)$$
		and
		\begin{eqnarray*}
			\nabla_{\theta_i}l &=& \sum_{k=1}^n\frac{1}{p(x_k|\theta)}\nabla_{\theta_i}\left(\sum_{j=1}^cp(x_k|\omega_j,\theta_j)P(\omega_j)\right)\\
			&=& \sum_{k=1}^n\frac{1}{p(x_k|\theta)}\nabla_{\theta_i}p(x_k|\omega_i)P(\omega_i)
		\end{eqnarray*}
		Here we assume that $\theta_i$  and $\theta_j$ are independent if $i\neq j$.
	\end{frame}
	\begin{frame}
		According the Bayes' formula,
		$$P(\omega_i|x_k,\theta)=\frac{p(x_k|\omega_i,\theta_i)P(\omega_i)}{p(x_k|\theta)}$$
		Hence,
		$$\nabla_{\theta_i}l=\sum_{k=1}^nP(\omega_i|x_k,\theta)\nabla_{\theta_i}\log p(x_k|\omega_i,\theta_i)$$
		and $P(\omega_i)$ can be estimate by $P(\omega_i|x_k)$:
		$$p(\omega_i)=\frac{1}{n}\sum_{k=1}^nP(\omega_i|x_k,\theta)$$ 
	\end{frame}
	\subsection{The EM algorithm}
	\begin{frame}
		\frametitle{Mixtures of Gaussians and the EM algorithm}
		Suppose that we are given a training set $\{x^{(1)},
		\ldots,x^{(m)}\}$, we wish to model the data by specifying
		a joint distribution $p(x^{(1)},z^{(i)})=p(x^{(i)}|z^{(i)})$.
		Here, 
		\begin{eqnarray*}
			z^{(i)}\in\{1,\ldots,k\}&\sim&\mathrm{Multinomial}(\phi),\\
			x^{(i)}|z^{(i)}=j&\sim&\mathcal{N}(\mu_j,\Sigma_j)
		\end{eqnarray*}
		This is call the \textbf{mixture of Gaussians} model, and
		$z^{(i)}$'s are \textbf{latent} random variables.
		To estimate $\phi$, $\mu$ and $\Sigma$, we can write down
		the likelihood of our data.
		\begin{eqnarray*}
			l(\phi,\mu,\Sigma) &=& \sum_{i=1}^m\log p(x^{(i)};
			\phi,\mu,\Sigma)\\
			&=& \sum_{i=1}^m\log\sum_{z^{(i)}=1}^kp(x^{(i)}|z^{(i)};
			\mu,\Sigma)p(z^{(i)};\phi) 
		\end{eqnarray*}		 
		
	\end{frame}
	\begin{frame}[shrink]
		\frametitle{Mixtures of Gaussians and the EM algorithm(2)}
		Note that if we knew what the $z^{(i)}$'s were, the maximum
		likelihood problem would have been easy. Specifically, we
		could then write down the likelihood as
		$$l(\phi,\mu,\Sigma)=\sum_{i=1}^m\log p(x^{(i)}|z^{(i)};\mu,
		\Sigma)+\log p(z^{(i)};\phi)$$
		Maximizing this with respect to $\phi$, $\mu$ and $\Sigma$ 
		gives the parameters.
		\begin{eqnarray*}
			\phi_j &=& \frac{1}{m}\sum_{i=1}^m1\{z^{(i)}=j\}\\
			\mu_j &=& \frac{\sum_{i=1}^m1\{x^{(i)}=j\}x^{(i)}}{
			\sum_{i=1}^m1\{z^{(i)}=j\}}\\
			\Sigma_j &=& \frac{\sum_{i=1}^m1\{z^{(i)}=j\}(x^{(i)}-
			\mu_j)(x^{(i)}-\mu_j)^T}{\sum_{i=1}^m1\{z^{(i)}=j\}}
		\end{eqnarray*}
	\end{frame}	
	\begin{frame}
		\frametitle{Mixtures of Gaussians and the EM algorithm(3)}
		The EM algorithm is an iterative algorithm that has two main
		step.
		\begin{figure}
           \pgfimage[width=9cm ]{"image/14.png"}
        \end{figure}
	\end{frame}	
	\begin{frame}
		\frametitle{Jensen’s inequality}
		\textbf{Theorem.} Let f be a convex(concave) function, and let X be
		a random variable. Then:
		$$E[f(X)]\ge f(EX)\quad\mbox{ (Concave case is $E[f(X)]\le f(EX)$ )}$$
		Moreover, if $f$ is strictly convex(concave), then $E[f(X)]=f(EX)$ 
		holds true if and only if $X=E[X]$ with probability 1.
		\begin{figure}
           \pgfimage[width=5cm ]{"image/15.png"}
        \end{figure}
	\end{frame}
	\begin{frame}[shrink]
		\frametitle{The EM algorithm}
		Suppose we have a training set $\{x^{(i)},\ldots,x^{(m)}\}$.
		We wish to fit the parameters of a model p(x,z) to the data,
		where the likelihood is given by
		$$l(\theta)=\sum_{i=1}^m\log p(x;\theta)=\sum_{i=1}^m
		\log\sum_zp(x,z;\theta).$$
		For each $i$, let $Q_i$ be some distribution over the $z$'s(
		$\sum_zQ_i(z)=1, Q_i(z)\ge 0$. Consider:
		\begin{eqnarray}
			\sum_i\log p(x^{(i)};\theta) &=& \sum_i\log\sum_{z^{(i)}}
			p(x^{(i)},z^{(i)};\theta)\\
			&=& \sum_i\log\sum_{z^{(i)}}Q_i(z^{(i)})\frac{p(x^{(i)},
			z^{(i)};\theta)}{Q_i(z^{i})}\\
			&=& \sum_i\log E_{z^{(i)}\sim Q_i}[\frac{p(x^{(i)},
			z^{(i)};\theta)}{Q_i(z^{(i)})}]\\
			&\ge& \sum_i\sum_{z^{(i)}}Q_i(z^{(i)})\log\frac{p(x^{(i)},
			z^{(i)};\theta)}{Q_i(z^{(i)})}
		\end{eqnarray}
	\end{frame}
	\begin{frame}[shrink]
		\frametitle{The EM algorithm(2)}
		To make the bound tight,we need Jensen's inequality to hold
		equality.we require that
		$$\frac{p(x^{(i)},z^{(i)})}{Q_i(z^{(i)})}=c$$
		for some constant c that not depend on $z^{(i)}$
		$$Q_i(z^{(i)})\propto p(x^{(i)},z^{(i)};\theta)$$
		Actually,since $\sum_zQ_i(z^{(i)})=1$:
		\begin{eqnarray*}
			Q_i(z^{(i)}) &=& \frac{p(x^{(i)},z^{(i)};\theta)}{
			\sum_zp(x^{(i)},z;\theta)}\\
			&=& \frac{p(x^{(i)},z^{(i)};\theta)}{p(x^{(i)};\theta)}=
			p(z^{(i)}|x^{(i)};\theta)
		\end{eqnarray*}
	\end{frame}	
	\begin{frame}[shrink]
		\frametitle{The EM algorithm(3)}
		Generalized EM algorithm:
		\begin{figure}
           \pgfimage[width=8cm ]{"image/16.png"}
        \end{figure}
        If we define
        $$J(Q,\theta)=\sum_i\sum_{z^{(i)}}Q_i(z^{(i)})\log\frac{
        p(x^{(i)},z^{(i)};\theta)}{Q_i(z^{(i)})}.$$
        the EM can also be viewed a coordinate ascent on J.
        $$$$
	\end{frame}
	\section{Graphical Model}
	\begin{frame}
		\frametitle{The envelope quiz}
		\begin{figure}
			
			\pgfimage[width=8cm ]{"image/22.png"}
			
		\end{figure}
		\begin{itemize}
			\item red ball is goal
			\item You randomly picked an envelop randomly took out a ball and it was black
			\item Should you choose this envelope or the other envelope?
		\end{itemize}
	\end{frame}
	\begin{frame}
		\frametitle{The envelope quiz(2)}
		\begin{itemize}
			\item Probabilistic inference
			\begin{itemize}
				\item Joint distribution on $E\in\{1,0\},B\in\{r,b\}$: $P(E,B)=P(E)P(B|E)$
				\item $P(E=1)=P(E=0)=\frac{1}{2}$
				\item $P(B=r|E=1)=\frac{1}{2},P(B=r|E=0)=0$
				\item The graphical model:
				\begin{figure}
					
					\pgfimage[width=1cm ]{"image/23.png"}
					
				\end{figure}
				
			\end{itemize}
			\item Statistical decision theory: switch if $P(E=1|B=b) < \frac{1}{2}$
			\item $P(E=1|B=b)=\frac{P(B=b|E=1)P(E=1)}{P(B=b)}=\frac{1}{3}$
		\end{itemize}
	\end{frame}
	\begin{frame}
		\frametitle{Some concepts}
		\begin{itemize}
			\item The world is reduced to a set of random variables $x_1,\ldots,x_d$
			\begin{itemize}
				\item e.g. $(x_1,\ldots,x_{d-1}) $ is a feature vector, $x_d\equiv y$ is the class label.
			\end{itemize}
			\item Inference: given joint distribution $p(x_1,\ldots,x_d),$ compute $p(X_Q|X_E)$ where $X_Q\cup X_E \subseteq \{x_1,\ldots,x_d\}$
			\begin{itemize}
				\item e.g. $Q=\{x_d\},E=\{x_1,\ldots,x_{d-1}\},$ by the definition of conditional
				$$p(x_d|x_1,\ldots,x_{d-1})=\frac{p(x_1,\ldots,x_d)}{\sum_vp(x_1,\ldots,x_{d-1},x_d=v)}$$
			\end{itemize}
			\item Learning estimate $p(x_1,\ldots,x_d)$ from training data $\{x^{(1)},\ldots,x^{(m)}\}$
		\end{itemize}
	\end{frame}
	\begin{frame}
		\frametitle{What are graphical models?}
		\begin{itemize}
			\item Graphical model = joint distribution $p(x_1,\ldots,x_d)$
			\begin{itemize}
				\item Bayesian network
				\item Markov random field
			\end{itemize}
			\item Inference = $p(X_Q|X_E)$, in general $X_Q\cup X_E \subseteq \{x_1,\ldots,x_d\}$
			\item If $p(x_1,\ldots,x_d) not given, estimate it from data$
			\begin{itemize}
				\item parameter and structure learning
			\end{itemize}
		\end{itemize}
	\end{frame}
	\begin{frame}
		\frametitle{What are graphical models?(2)}
		\begin{itemize}
			\item Graphical model is the study of probabilistic models
			\item Just because there are nodes and edges doesn't mean it's a graphical model
			\item These are not graphical models:
			\begin{figure}
				
				\pgfimage[width=10cm ]{"image/24.png"}
				
			\end{figure}
		\end{itemize}
	\end{frame}
	\begin{frame}
		\frametitle{Directed graphical models}
		\begin{itemize}
			\item Also called Bayesian networks
			\item A directed graph has nodes $x_1,\ldots,x_d$, some of them connected by directed edges $x_i\rightarrow x_j$
			\item A cycle is a directed path $x_1\rightarrow\ldots\rightarrow x_k$ where $x_1=x_k$
			\item A directed acyclic graph (DAG) contains no cycles 
		\end{itemize}
	\end{frame}
	\begin{frame}
		\frametitle{Directed graphical models(2)}
		\begin{itemize}
			\item A Bayesian network on the DAG is a family of distributions satisfying
			$${p|p(x_1,\ldots,x_d)=\prod_ip(x_i|Pa(x_i))}$$
			where $Pa(x_i)$ is the set of parents of $x_i$
			\item $p(x_i|Pa(x_i))$ is the conditional probability distribution (CPD) at $x_i$
			\item By specifying the CPDs for all $i$, we specify a joint distribution $p(x_1,\ldots,x_d)$
		\end{itemize}
	\end{frame}
	\begin{frame}
		\frametitle{Example:Burglary, Earthquake, Alarm, John and Marry}
		\begin{figure}
			
			\pgfimage[width=8cm ]{"image/25.png"}
			
		\end{figure}
		$$P(B,\sim E,A,J,\sim M) = P(B)P(\sim E)P(A|B,\sim E)P(J|A)P(\sim M|A)$$
	\end{frame}
	\begin{frame}
		\frametitle{Example: Naive Bayes}
		\begin{figure}
			
			\pgfimage[width=8cm ]{"image/26.png"}
			
		\end{figure}
		\begin{itemize}
			\item $p(y,x_1,\ldots,x_d)=p(y)\prod_{i=1}^{d}p(x_i|y)$
			\item Plate representation on the right
			\item $p(y)$ multinomial
			\item $p(x_i|y)$ depends on the feature type.
		\end{itemize}
	\end{frame}
	\begin{frame}
		\frametitle{Undirected graphical models}
		\begin{itemize}
			\item Also known as Markov Random Fields
			\item A clique C in an undirected graph is a set of fully connected nodes (full of loops!)
			\item Define a nonnegative potential function $\psi_C:X_C\rightarrow \mathbb{R_+}$
			\item An undirected graphical model is a family of distributions satisfying
			$$\left\{p|p(X)=\frac{1}{Z}\prod_C\psi_C(X_c)\right\}$$
			\item $Z=\int\prod_C\psi_C(X_C)dX$
		\end{itemize}
	\end{frame}
	\begin{frame}
		\frametitle{A Tiny Markov Random Field}
		\begin{figure}
			
			\pgfimage[width=5cm ]{"image/27.png"}
			
		\end{figure}
		\begin{itemize}
			\item $x_1,x_2\in \{-1,1\}$
			\item A single clique $\psi_C(x_1,x_2)=e^{ax_1x_2}$
			\item $p(x_1,x_2) = \frac{1}{Z}e^{ax_1x_2}$
			\item $Z=(2e^a+2e^{-a})$
			\item $p(1,1)=p(-1,-1)=\frac{e^a}{2e^a+2e^{-a}}$
		\end{itemize}
	\end{frame}
	\section{Topic Model}
	\begin{frame}
		\begin{itemize}
			\frametitle{Topic Model}
			\item A topic is defined as a probability distribution over terms or a cluster of weighted terms.
			\item A document is defined as a set of words generated from a mixture of latent topics.
		\end{itemize}
		Various topic modeling methods, such as PLSI, LDA, LSI, NMF, and RLSI have been proposed and successfully applied to different applications.
	\end{frame}
	\begin{frame}
		\frametitle{Probabilistic Topic Models}
		Suppose that $\mathcal{D}=\{d_1,d_2,\ldots,d_N\}$ is a set of documents with size $N$ and $\mathcal{V}$ is a set of terms or words with size $M$, i.e., the vocabulary. A document $d\in\mathcal{D}$ consists of $|d|$ words from the vocabulary, denoted as $d=(w_1,w_2,\ldots,w_{|d|})$.
		Suppose that there are $K$ topics in the document collection.
		
		PLSI is one of the widely used probabilistic topic models. One can generate the documents in the collection in the following way.
		\begin{itemize}
			\item select a document $d$ from the collection with probability $P(d)$
			\item select a latent topic $z$ with probability $P(z|d)$
			\item generate a word $w$ with probability $P(w|z)$
		\end{itemize}
		Here $z\in \{z_1,\ldots,z_K\}$ is a latent variable representing a topic.
		
		
	\end{frame}	
	\begin{frame}
		\frametitle{Probabilistic Topic Models(2)}
		The parameters of $P(d)$, $P(w|z)$, and $P(z|d)$ can be estimated by EM algorithm.
		\begin{figure}
			\pgfimage[width=6cm ]{"image/19.png"}
		\end{figure}
	\end{frame}
	\begin{frame}
		\frametitle{Probabilistic Topic Models(3)}
		LDA Model:
		\begin{figure}
			\pgfimage[width=6cm ]{"image/20.png"}
		\end{figure}
		\begin{itemize}
			\item[1.] for each topic $k=1,\ldots,K$
			%\subitem[(a)] draw 
			\begin{itemize}
				\item[(a)] draw word distribution $\phi_k$ according to $\phi_k|\beta\sim\mathrm{Dir}(\beta)$
			\end{itemize}
			\item[2.] for each document $d$ in the collection
			\begin{itemize}
				\item[(a)] draw topic distribution $\theta$ according to $\theta|\alpha\sim\mathrm{Dir}(\alpha)$
				\item[(b)] for each word w in the document d
				\begin{itemize}
					\item[i.] draw a topic $z$ according to $z|\theta\sim\mathrm{Mult}(\theta)$
					\item[ii.] draw a word $w$ according to $w|z\sim\mathrm{Mult}(\phi_z)$
				\end{itemize}
			\end{itemize}
		\end{itemize}
	\end{frame}
	\begin{frame}
		\frametitle{Non-probabilistic Topic Models}
		Non-probabilistic topic models are usually obtained by matrix factorization.Suppose that $\mathcal{D}$ is a set of document with size $N$, and $\mathcal{V}$ is a vocabulary with size $M$.The document collection $\mathcal{D}$ is represented as an $M\times N$ matrix $D=(d_1,\ldots,d_N)$. 
		
		Suppose that there are $K$ topics represented as an $M\times K$ term-topic matrix $U=(u_1,\ldots,u_K)$. $V^T=(v_1,\ldots,v_N)$ is an $K\times N$ topic-document matrix.
		
		Our goal is to factor matrix $D$ by $U$ and $V^T$.
		$$D\approx UV^T$$
	\end{frame}
	\begin{frame}
		\frametitle{Non-probabilistic Topic Models(2)}
		LSI assumes that the $K$ columns of matrix $U$ as well as the $K$ columns of matrix $V$ are orthonormal. LSI amounts to minimizing the following objective function with the orthonormal constraints.
		
		$$ \min_{U,V}\|D-U\Sigma V^T\|_F$$
		$$ s.t.\quad U^T \times U=I, V\times V^T=I, \mbox{and $\Sigma$ is diagonal} $$
		\begin{figure}
			\pgfimage[width=8cm ]{"image/21.png"}
		\end{figure}

	\end{frame}
	\section{Maximum Entropy Models}
	\begin{frame}
		\frametitle{What is entropy}
		In thermodynamics, entropy (usual symbol $S$) is a measure of the number of specific ways in which a thermodynamic system may be arranged, commonly understood as a measure of disorder.
		$$\Delta = \int\frac{dQ_{rev}}{T}$$
		In information theory, \textbf{Entropy} is a measure of unpredictability of information content. Shannon defined the entropy $H$ of a discrete random variable $X$ with possible values $x1, ..., xn$ and probability mass function P(X) as:
		$$H(X) = E[I(X)] = E[-\ln(P(X))]$$
		Here $I$ is the information content of X.
		
	\end{frame}
	
	\begin{frame}
		\frametitle{What is entropy(2)}
		When taken from a finite sample, the entropy can explicitly be written as
		$$H(X) = \sum_i P(x_i)I(x_i) = -\sum_i P(x_i)\log_b P(x_i)$$
		Common values of $b$ are 2, $e$, and 10, and the unit of entropy is bit for $b = 2$, nat for $b = e$, and dit (or digit) for $b = 10$.
		\begin{figure}
			\pgfimage[width=4cm ]{"image/28.png"}
		\end{figure}
		
	\end{frame}
	
	\begin{frame}[shrink]
		\frametitle{Whis is entropy(3)}
		According the definition of $H(X)$, we can define \textbf{Joint Entropy} as:
		$$H(X,Y) = -\sum_{x,y}p(x,y)\log p(x,y)$$
		and  define \textbf{Conditional Entropy} to measure unpredictability of random variable $Y$ given random variable $X$ as:
		$$H(Y|X) = -\sum_{x,y} p(x,y)\log p(y|x)$$
		The entropy or the amount of information revealed by evaluating $(X,Y)$ (that is, evaluating $X$ and $Y$ simultaneously) is equal to the information revealed by conducting two consecutive experiments: first evaluating the value of $Y$, then revealing the value of $X$ given that you know the value of $Y$. This may be written as
		\begin{equation}
			H(X,Y) = H(X|Y)+H(Y)=H(Y|X)+H(X)
		\end{equation}	
	\end{frame}
	
	\begin{frame}[shrink]	
		\frametitle{Whis is entropy(4)}
		We can get (7) by:
		\begin{eqnarray*}
			&& H(X,Y)-H(X) \\
			&=& -\sum_{x,y}p(x,y)\log p(x,y) + \sum_x p(x)\log p(x)\\
			&=& -\sum_{x,y}p(x,y)\log p(x,y) + \sum_x\left(\sum_y p(x,y)\right)\log p(x)\\
			&=& -\sum_{x,y}p(x,y)\log p(x,y)+ -\sum_{x,y}p(x,y)\log p(x)\\
			&=& -\sum_{x,y}p(x,y)\log\frac{p(x,y)}{p(x)}\\
			&=& -\sum_{x,y} p(x,y)\log p(y|x) = H(Y|X)
		\end{eqnarray*}

	\end{frame}
	
	\begin{frame}
		\frametitle{Whis is entropy(5)}
		Another useful measure of entropy that works equally well in the discrete and the continuous case is the \textbf{Relative Entropy} of a distribution. It is also called as the \textbf{Kullback-Leibler Divergence}.
		$$D(p||q) = \sum_x p(x)\log \frac{p(x)}{q(x)} = E_{p(x)}\log\frac{p(x)}{q(x)}$$
		To some extent, Relative Entropy can be the measure of the distance between two random variables.
		
		\textbf{Mutual Information} can be defined as:
		$$I(X,Y)=\sum_{x,y}\log\frac{p(x,y)}{p(x)p(y)}$$
	\end{frame}
	
	\begin{frame}[shrink]
		\frametitle{What is entropy(6)}
		Let's consider
		\begin{eqnarray*}
			&& H(Y)-I(X,Y)  \\
			&=& -\sum_y p(y)\log p(y) - \sum_{x,y}\log\frac{p(x,y)}{p(x)p(y)} \\
			&=& -\sum_y\left(\sum_x p(x,y)\right)\log p(y) - \sum_{x,y}p(x,y)\log\frac{p(x,y)}{p(x)p(y)} \\
			&=& -\sum_{x,y}p(x,y)\log p(y) - \sum_{x,y}\log\frac{p(x,y)}{p(x)p(y)} \\
			&=& -\sum_{x,y}p(x,y)\log\frac{p(x,y)}{p(x)} \\
			&=& -\sum_{x,y}p(x,y)\log p(y|x) = H(Y|X)
		\end{eqnarray*}
		According the formula (7):
		$$I(X,Y) = H(X)+H(Y)-H(X,Y)$$
	\end{frame}
	
	\begin{frame}[shrink]
		\frametitle{Motivating example}
		Suppose we want a model translate English word \textit{in} to French.
		
		Let $p$ denote the model, and $p(f)$ denote the probability that $f$ is chose as a translation of \textit{in}.($f \in \{\mbox{dans,en,\`{a},au cours de,pendant}\}$)
		
		Now, we can impose our first constraint on our model $p$:
		$$p(\mbox{dans})+p(\mbox{en})+p(\mbox{\`a})+p(\mbox{au cours de})+p(\mbox{pendant}) = 1$$
		With no empirical justification, the most in intuitively model is
		$$p(\mbox{dans})=p(\mbox{en})=p(\mbox{\`a})=p(\mbox{au cours de})=p(\mbox{pendant})=\frac{1}{5}$$
		If we add an additional constraint
		$$p(\mbox{dans})+p(\mbox{en}) = \frac{1}{3}$$
		subject to the constraints
		$$p(\mbox{dans})=p(\mbox{en})=\frac{3}{20};p(\mbox{\`a})=p(\mbox{au cours de})=p(\mbox{pendant})=\frac{7}{30} $$
		
	\end{frame}
	
	\begin{frame}
		\frametitle{Training data}
		Let's do some form formalization, suppose we have collected a large number of samples $\{(x_1,y_1),(x_2,y_2),\ldots,(x_N,y_N)\}$.
		\newline
		\newline
		In the example we have been considering, each sample would consist of a phrase $x$ containing the words surrounding \textit{in}, together with the translation $y$ of \textit{in}.
		\newline
		\newline
		We can summarize the training example in terms of its \textbf{empirical probability distribution} $\widetilde{p}$,define by
		$$\widetilde{p}(x,y) = \frac{1}{N} \times \mbox{number of times that $(x,y)$ occurs in sample}$$
	\end{frame}
	
	\begin{frame}
		\frametitle{Features and constraints}
		First, let's consider the definition of \textbf{feature function}
		$$f(x,y)=\left\{
		\begin{array}{ll}
		1  &\mbox{if $x,y$ satisfy some condition}\\
		0  &\mbox{otherwise}
		\end{array}
		\right.
		$$
		In our example we have employed some constraints, but we could also consider statistics which depend on $x$.
		
		For instance, we might notice that, in the training example, if \textit{April} is the word following \textit{in}, then the translation of in is en with frequency $\frac{9}{10}$.
		
		Now we can use feature function to express the event that \textit{in} translates as \textit{en} when \textit{April} is the following word.
		$$f(x,y)=\left\{
		\begin{array}{ll}
		1  &\mbox{if $y=en$ and \textit{April} follows \textit{in}}\\
		0  &\mbox{otherwise}
		\end{array}
		\right.
		$$
	\end{frame}
	
	\begin{frame}
		\frametitle{Features and constraints(2)}
		The expected value of $f$ with respect to the empirical distribution $\widetilde{p}(x,y)$
		\begin{equation}
			\widetilde{p}(f) = \sum_{x,y}\widetilde{p}(x,y)f(x,y)
		\end{equation}
		The expected value of $f$ with respect to real distribution $p(x,y)$
		\begin{equation*}
			p(f) = \sum_{x,y}p(x,y)f(x,y) = \sum_{x,y}p(x)p(y|x)f(x,y)
		\end{equation*}
		For convenience, we usually use $\widetilde{p}(x)$ instead of $p(x)$
		\begin{equation}
		p(f) = \sum_{x,y}\widetilde{p}(x)p(y|x)f(x,y)
		\end{equation}
	\end{frame}
	
	\begin{frame}
		\frametitle{Features and constraints(3)}
		We constrain this expected value to be the same as the expected value of f in the training sample. That is, we require
		\begin{equation}
			p(f) = \widetilde{p}(f). 
		\end{equation}
		Combining (8), (9) and (10), we obtain
		\begin{equation}
			\sum_{x,y}\widetilde{p}(x)p(y|x)f(x,y) = \sum_{x,y}\widetilde{p}(x,y)f(x,y).
		\end{equation}
		We call the requirement (11) a \textbf{constraint equation} or simply a \textbf{constraint}.
	\end{frame}
	
	\begin{frame}
		\frametitle{The Maximum Entropy Principle}
		Suppose that we are given $n$ feature function $\{f_1,\ldots,f_n\}$, which determine statistics we feel are import in modeling the process.
		
		We would like our model to accord with these conditions. That is, we would like $p$ to lie in the subset $\mathcal{C}$ of $\mathcal{P}$ defined by
		\begin{equation}
			\mathcal{C} = \left\{ p\in\mathcal{P} | p(f_i) = \widetilde{p}(f_i) \quad i\in \{1,2,\ldots,n\} \right\}
		\end{equation}
		\begin{figure}
			\pgfimage[width=6cm ]{"image/31.png"}
		\end{figure}
	\end{frame}
	
	\begin{frame}
		\frametitle{The Maximum Entropy Principle(2)}
		Among the models $p\in \mathcal{C}$, the maximum entropy philosophy dictates that we select the distribution which is most uniform, which means we choose $p$ maximizing the entropy
		\begin{equation}
			H(p) = -\sum_{x,y}\widetilde{p}(x,y)p(y|x)\log p(y|x)
		\end{equation}
		With these definition in hand, we can present the principle of maximum entropy
		\begin{equation}
			p^* = \mathrm{arg}\max_{p\in \mathcal{C}} H(p)
		\end{equation} 
		It can be shown that $p^*$ is well defined; that is, there is always a unique model $p^*$ with maximum entropy in any constrained set $\mathcal{C}$ 
	\end{frame}

	\begin{frame}
		\frametitle{Optimization problem}
		The constrained optimization problem at hand is to find
		\begin{equation}
			p^* = \mathrm{arg}\max_{p\in\mathcal{C}}\left(-\sum_{x,y}\widetilde{p}(x)p(y|x)\log p(y|x) \right)
		\end{equation}
		subject to
		\begin{eqnarray}
			&& p(y|x) \geq 0 \quad \mbox{for all $x,y$}\\
			&& \sum_y p(y|x) = 1 \quad \mbox{for all $x$}\\
			&& \widetilde{p}(f_i) = p(f_i) \quad \mbox{for $i\in \{1,2,\ldots,n\}$}
		\end{eqnarray}
	\end{frame}
	
	\begin{frame}
		\frametitle{Optimization problem(2)}
		To solve this optimization problem, introduce the Lagrangian
		\begin{equation}
			\mathcal{L}(p,\Lambda) = -H(p) + \lambda_0\left(1-\sum_y p(y|x) \right) + \sum_{i=1}^{n}\lambda_i\left(\widetilde{p}(f_i)-p(f_i)\right),
		\end{equation}
		where $\Lambda = \{\lambda_0,\lambda_1,\ldots,\lambda_n\}$.
		
		Then we obtain the \textbf{primary problem}
		\begin{equation}
			\min_{p\in\mathcal{C}}\max_{\Lambda}\mathcal{L}(p,\Lambda)
		\end{equation}
		For $-H(p)$ is convex function, the primary problem is equivalent to
		its \textbf{dual problem}
		\begin{equation}
			\max_{\Lambda}\min_{p\in\mathcal{C}}\mathcal{L}(p,\Lambda)
		\end{equation}
	\end{frame}
	
	\begin{frame}
		\frametitle{Exponential form}
		Consider the minimum problem in dual problem:
		\begin{equation}
			\Psi(\Lambda) = \min_{p\in\mathcal{C}}\mathcal{L}(p,\Lambda) = \mathcal{L}(p_\Lambda,\Lambda),
		\end{equation}
		where
		\begin{equation}
			p_\Lambda = \mathrm{arg}\max_{p\in\mathcal{C}}\Lambda(p,\Lambda).
		\end{equation}
		
		
	\end{frame}
		
	\begin{frame}
		\frametitle{Exponential form(2)}
		The we use \textbf{Lagrangian multiplier method} to obtain $p_\Lambda$
		\begin{eqnarray*}
			&&\frac{\partial\mathcal{L}(p,\Lambda)}{\partial p(y|x)} \\
			&=& \sum_{x,y}\widetilde{p}(x)(\log p(y|x) +1) - \sum_y\lambda_0 - \sum_{i=1}^{n}\left(\sum_{x,y}\widetilde{p}(x)f_i(x,y)\right)\\
			&=& \sum_{x,y}\widetilde{p}(x)(\log p(y|x) +1) - \sum_x\widetilde{p}(x)\sum_y\lambda_0 - \sum_{x,y}\widetilde{p}(x)\sum_{i=1}^{n}\lambda_i f_i(x,y)\\
			&=& \sum_{x,y}\widetilde{p}(x)(\log p(y|x) +1) - \sum_{x,y}\widetilde{p}(x)\lambda_0 - \sum_{x,y}\widetilde{p}(x)\sum_{i=1}^{n}\lambda_i f_i(x,y)\\
			&=& \sum_{x,y}\left(\log p(y|x) +1 -\lambda_0 -\sum_{i=1}^{n}\lambda_i f_i(x,y) \right)
		\end{eqnarray*}
	\end{frame}
	
	\begin{frame}
		\frametitle{Exponential form(3)}
		Let $\frac{\partial\mathcal{L}(p,\Lambda)}{\partial p(y|x)}$ be zero, we obtain
		\begin{equation}
			\log p(y|x) +1 -\lambda_0 -\sum_{i=1}^{n}\lambda_i f_i(x,y) = 0,
		\end{equation}
		then
		\begin{equation}
			p(y|x) = e^{\lambda_0-1}\cdot e^{\sum_{i=1}^{n}\lambda_i f_i(x,y)}.
		\end{equation}
		According to constraint (17), we obtain
		\begin{equation}
			p_\Lambda = \frac{1}{Z_\Lambda(x)}e^{\sum_{i=1}^{n}\lambda_i f_i(x,y)}
		\end{equation}
		where
		\begin{equation}
			Z_\Lambda(x) = \sum_y e^{\sum_{i=1}^{n}\lambda_i f_i(x,y)}
		\end{equation}
	\end{frame}
	
	\begin{frame}[shrink]
		\frametitle{Maximum likelihood}
		Consider the maximum problem in dual problem:
		\begin{equation}
			\max_\Lambda \Psi(\Lambda)
		\end{equation}
		We suppose
		\begin{equation}
			\Lambda^* = \mathrm{arg}\max_\Lambda \Psi(\Lambda),
		\end{equation}
		then we can solve the optimization problem by $p^* = p_\Lambda$.
		
		We know
		\begin{eqnarray*}
			\Psi(\Lambda) &=& \mathcal{L}(p_\Lambda,\Lambda)\\
			&=&\sum_{x,y}\widetilde{p}(x)p_\Lambda(y|x)\log p_\Lambda(y|x) \\
			&+& \sum_{i=1}^{n}\lambda_i\left(\widetilde{p}(f_i) - \sum_{x,y}\widetilde{p}(x)p_\Lambda(y|x)f_i(x,y)\right)\\
			&=& \sum_{i=1}^{n} \lambda_i\widetilde{p}(f_i) +\sum_{x,y}\widetilde{p}(x)p_\Lambda(y|x)\left(\log p_\Lambda(y|x) - \sum_{i=1}^{n}\lambda_if_i(x,y)\right)
		\end{eqnarray*}
	\end{frame}
	
	\begin{frame}
		\frametitle{Maximum likelihood (2)}
		According to (26), we obtain
		\begin{equation}
			\log p_\Lambda(y|x) = \sum_{i=1}^{n}\lambda_i f_i(x,y) - \log Z_\Lambda(x)
		\end{equation}
		substitute it to $\Psi(\Lambda)$,
		\begin{eqnarray*}
			\Psi(\Lambda) &=& \sum_{i=1}^{n}\lambda_i\widetilde{p}(f_i)-\sum_{x,y}\widetilde{p}(x)p_\Lambda(y|x)\log Z_\Lambda(x) \\
			&=& \sum_{i=1}^{n}\lambda_i\widetilde{p}(f_i) - \sum_x \widetilde{p}(x)\log Z_\Lambda(x)\sum_y p_\Lambda(y|x)\\
			&=& \sum_{i=1}^{n}\lambda_i\widetilde{p}(f_i) - \sum_x\widetilde{p}(x)\log Z_\Lambda(x)
		\end{eqnarray*}
	\end{frame}
	
	\begin{frame}[shrink]
		\frametitle{Maximum likelihood (3)}
		Consider MLE:
		\begin{equation}
			\mathcal{L}_{\widetilde{p}}(p) = \log \prod_{x,y} p(y|x)^{\widetilde{p}(x,y)} = \sum_{x,y}\widetilde{p}(x,y)\log p(y|x)
		\end{equation}
		According (30) we obtain,
		\begin{eqnarray*}
			\mathcal{L}_{\widetilde{p}}(p_\Lambda) &=& \sum_{x,y}\widetilde{p}(x,y)\left(\sum_{i=1}^{n} \lambda_if_i(x,y) - \log Z_\Lambda(x) \right)\\
			&=& \sum_{x,y}\widetilde{p}(x,y)\sum_{i=1}^{n}\lambda_if_i(x,y)-\sum_{x,y}\widetilde{p}(x,y)\log Z_\Lambda(x) \\
			&=& \sum_{i=1}^{n}\lambda_i\left(\sum_{x,y}\widetilde{p}(x,y)f_i(x,y)\right) - \sum_{x,y}\widetilde{p}(x,y)\log Z_\Lambda(x)\\
			&=& \sum_{i=1}^{n}\lambda_i\widetilde{p}(f_i) - \sum_x\widetilde{p}(x)\log Z_\Lambda(x)
		\end{eqnarray*}
	\end{frame}
	
	\begin{frame}
		\frametitle{Computing the Parameters}
		Some Parameters Optimization method in common use like\textbf{ Gradient descent}:
		\begin{eqnarray*}
			\frac{\partial\Psi}{\partial\lambda_i} &=& \frac{\partial}{\partial\lambda_i}\left(\sum_{i=1}^{n}\lambda_i\widetilde{p}(f_i) - \sum_x\widetilde{p}(x)\log Z_{\Lambda}(x) \right)\\
			&=& \widetilde{p}(f_i) - \sum_x\widetilde{p}(x)\frac{1}{Z_{\Lambda}(x)}\frac{\partial}{\partial\lambda_i}\left(\sum_y e^{\sum_{i=1}^{n}\lambda_if_i(x,y)} \right)\\
			&=& \widetilde{p}(f_i) - \sum_x\widetilde{p}(x)\frac{1}{Z_{\Lambda}(x)}\sum_y e^{\sum_{i=1}^{n}\lambda_if_i(x,y)}f_i(x,y)\\
			&=& \widetilde{p}(x) - \sum_x \widetilde{p}(x)\sum_y p_{\Lambda}(y|x)f_i(x,y)
		\end{eqnarray*}
	\end{frame}
	
	\begin{frame}
		\frametitle{GIS algorithm}
		\begin{itemize}
			\item[Step1] Initial parameters. set $\Lambda = 0$
			\item[Step2] Calculate $E_{\widetilde{p}(f_i)} = \sum_{i=1}^{n}\widetilde{p}(x,y)f_i(x,y)$, for $i=1,2,\ldots,n.$
			\item[Step3] Iteration, update parameters:
			\begin{itemize}
				\item[] Calculate $E_{p_{\Lambda}}(f_i)$, $i=1,2,\ldots,n$.
				\item[] \textbf{FOR} $i=1,2,\ldots,n$ \textbf{DO}
				\item[] \{
				\item[] $\lambda_i := \lambda_i + \eta\log\frac{E_{\widetilde{p}}(f_i)}{E_{p_{\Lambda}}(f_i)}$
				\item[] \}
			\end{itemize}
			\item[Step4] Check convergence.
		\end{itemize} 	
	\end{frame}
	
	\begin{frame}
		\frametitle{IIS algorithm}
		\begin{itemize}
			\item[Step1] Initial parameters, set $\Lambda := 0$
			\item[Step2] Iteration, update parameters:
			\begin{itemize}
				\item[] \textbf{FOR} $i=1,2,\ldots,n$ \textbf{DO}
				\item[] \{
				\item[] 	Solve equations
				\item[]    $\sum_{x,y}\widetilde{p}(x)p(y|x)f_i(x,y)e^{\delta_i\sum_{i=1}^{n}f_i(x,y)} = \widetilde{p}(f_i)$
				\item[] Let $\lambda_i := \lambda_i + \delta_i$
				\item[] \}
			\end{itemize}
			\item[Step3] Check convergence.
		\end{itemize}
	\end{frame}
	
	\section{Distance and Divergence}
	\begin{frame}
		\frametitle{Distance}
		\textbf{Distance} satisfies four constraints which \text{Divergence} may no satisfy.
		\begin{itemize}
			\item \textbf{Non-negativity:} $\forall x,y,\quad d(x,y) \ge 0$
			\item \textbf{Non-degeneracy:} $d(x,y)=0 \Leftrightarrow x=y$
			\item \textbf{Symmetry:} $\forall x,y, \quad d(x,y)=d(y,x)$
			\item \textbf{Triangularity:} $\forall x,y,z \quad d(x,z)\leq d(x,y)+d(y,z)$
		\end{itemize}
		\begin{figure}
			\pgfimage[width=4cm]{"image/29.png"}
		\end{figure}
			
	\end{frame}
		
	\begin{frame}
		\frametitle{Distance(2)}
		In the \textbf{Euclidean space} $\mathbb{R}^n$ ,the distance between two point can be given by \textbf{Minkowski distance}.(When p=2, it is equivalent the \textbf{Euclidean distance} and when p=1, it is equivalent to \textbf{Manhattan distance} ) Other distance, based on norms, are sometimes used instead.
		\begin{itemize}
			\item 1-norm distance  $= \sum_{i=1}^{n}|x_i-y_i|$
			\item 2-norm distance  $= \left(\sum_{i=1}^{n}|x_i-y_i|^2\right)^{1/2}$
			\item p-norm distance  $=  \left(\sum_{i=1}^{n}|x_i-y_i|^p\right)^{1/p}$
			\item $\infty$ norm distance $=\lim\limits_{p\rightarrow\infty}\left(\sum_{i=1}^{n}|x_i-y_i|^p\right)^{1/p}$
		\end{itemize}
		If we care about the direction of the data rather than the magnitude, then using the \textbf{cosine distance} is a common approach. 
		$$\cos \theta = \frac{A\cdot B}{\|A\|\|B\|}$$
	\end{frame}
	
	\begin{frame}
		\frametitle{Distance(3)}
		\textbf{Why is Euclidean distance not a good metric in most situation?}
		\begin{itemize}
			\item Vector length often is not determinant.
			\item Most Problems are in high-dimensional.
			\item Most of the volume of a high-dimensional orange is in the skin, not the pulp.
		\end{itemize}
		\begin{figure}
			\pgfimage[width=3cm ]{"image/30.png"}
		\end{figure}
	\end{frame}
	
	\begin{frame}
		\frametitle{Divergence}
	\end{frame}
\end{document}

